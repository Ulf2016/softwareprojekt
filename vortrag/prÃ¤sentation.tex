%\documentclass[aspectratio=169]{beamer}
\documentclass{beamer}
% \usepackage{beamerthemesplit} // Activate for custom appearance
\usetheme {Dresden}
\usepackage[utf8]{inputenc}
\usepackage{etoolbox}


\usenavigationsymbolstemplate{}

\addtobeamertemplate{footline}{centered}
{%
   \usebeamercolor[fg]{author in sidebar}
   \vskip-1cm
\hspace*{\fill} Spezifikationsvortrag Softwareprojekt\hspace*{\fill}%
\llap
   \insertframenumber\,/\,\inserttotalframenumber\kern1em\vskip2pt%
}

\begin{document}

\title[]{Spezifikationsvortrag}
\author[]{Fabian Düker, Uli Steinbach}
\institute{Universität Heidelberg, Institut für Computerlinguistik\linebreak\vspace{0.03\textwidth}
Softwareprojekt, SoSe 2018\linebreak\vspace{0.02\textwidth}
Prof Dr. Katja Markert}
\date{12.06.2018}

\frame{\titlepage}
\begin{frame}[allowframebreaks]{Übersicht}
\tableofcontents
\end{frame}


%%%%%%%%%%%%%%%%%%%% SECTION 1 %%%%%%%%%%%%%%%%%%%%%%%%%%%%%%%%%%%%%%%%%%%%%%%%%%%%%%%%%%%%%%%%%%%%%%%%%%%%%%%

\section{Übersicht}

\subsection[Übersicht]{Übersicht }

\begin{frame}{Übersicht}
\begin{block}{Autom. Erstellung eines Lexikons für die Erkennung von Abusive Words }
Anwendung auf Germeval Task I 2018 \\Binäre Klassifikation von 5000 Tweets 
\end{block}
\end{frame}


\section{inhaltliche Spezifikation}
\subsection[inh. Spez.]{ inh. Spezifikation }

\begin{frame}{Problemstellung}
\begin{itemize}
	\item Problem: Hatespeech ist in ständiger Veränderung begriffen (Neologismen, Ambiguität, Kontext)
	\item Wiegand et al. 2016: Erstellung eines englischen Lexikons mit guten Ergebnissen auf cross-domain Evaluation
	\item SentiWS: Lexikon mit negativen Wörtern für das Deutsche

\end{itemize}
\end{frame}

\begin{frame}{Lösungsansatz}
\begin{itemize}
\item Erstellung Baselexikon aus SentiWS neg. Sentiment-Lexikon
\item halbautomatische Erweiterung des Baselexikons mit deutschen Schimpwörtern
\item autom. Erweiterung mittels graphbasiertem Label-Propagation-Algorithmus
\item Anwendung auf Germeval 2018 Datenset und Evaluation
\end{itemize}
\end{frame}

\begin{frame}{halbautom. Erweiterung mit deutschen Schimpfwörtern}
\begin{itemize}
\item Genius API: Erstellung eines Deutschrapkorpus 
\item Deutschrap: zeitgemäße Verwendung von Schimpfwörtern (genrespezifisch, aber auch politisch + rassistisch)
\item autom. Extraktion von Kandidaten mittels syntaktischer Pattern
\item Beispielpattern: Du [NN] , Du [ADJ]* [NN]
\item manuelle Bereinigung der extrahierten Daten und Auswahl von Schimpfwörtern
\end{itemize}
\end{frame}

\begin{frame}{Auszug aus Korpus}
\begin{itemize}
\item Liste mit Songs/Artists, Textauszug, Übersicht Top-Schimpfwörter
\end{itemize}
\end{frame}


\begin{frame}{Evaluation}
\begin{itemize}
\item Baseline 1: Unigram und Bigram SVM 
\item Baseline 2: Feature Selection (Mutual Information) SVM 
\end{itemize}
\end{frame}

\begin{frame}{Evaluation}
\begin{itemize}
\item Tabelle Baseline 1+2
\item Tabelle Baseline 3
\end{itemize}
\end{frame}


\section{Modularisierung und Aufgabenverteilung}
\subsection[Modularisierung]{ Modularisierung und Aufgabenverteilung }

\begin{frame}{Aufgabenverteilung}
\end{frame}

\begin{frame}{Zeitplan}
\end{frame}


\section{Programmarchitektur, Datenstrukturen}
\subsection[Programmarchitektur]{ Programmarchitektur und Datenstrukturen }

\begin{frame}{Programmarchitektur}
\end{frame}

\begin{frame}{Datenstrukturen}
\end{frame}

\section{}
\subsection[]{}


%%%%%%%%%%%%%%%%%%%% SECTION %%%%%%%%%%%%%%%%%%%%%%%%%%%%%%%%%%%%%%%%%%%%%%%%%%%%%%%%%%%%%%%%%%%%%%%%%%%%%%%

\begin{frame}{Literatur}
\end{frame}



\end{document}
